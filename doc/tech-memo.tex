\documentclass[11pt]{article}
\usepackage{mathptmx}
\usepackage[tmargin=1.25in,bmargin=1.25in,hmargin=1.3in]{geometry}
\usepackage{graphicx}
\usepackage{fancyhdr}
\usepackage[usenames,dvipsnames,svgnames,table]{xcolor}
\usepackage{xspace}
\usepackage[hyphens]{url}
\usepackage{wrapfig}
% \usepackage{hyperref}

\newcommand{\doctitle}{Documentation of technology decisions for AMS-102}
\newcommand{\docauthor}{Tom Sgouros Brown/CCV}
\newcommand{\docdate}{March 5, 2018}

\pagestyle{fancy}
\lhead{\itshape\doctitle}%
%\chead{\raisebox{\baselineskip}{\@memohead}}
\rhead{\itshape\docdate}
\lfoot{\itshape\docauthor}
\cfoot{}
\rfoot{\thepage}
\renewcommand{\headrulewidth}{0.4pt}
\renewcommand{\footrulewidth}{0.4pt}

\begin{document}
\title{\doctitle}\author{\docauthor}\date{\docdate}
\maketitle

The goal is to make a visualization pipeline for hydrodynamic model data
that is easy to use on a portable device like a tablet.  The idea is
to make the visualizations available on the shop floor to
users relatively unsophisticated in graphics and visualization.

\begin{center}\setlength{\fboxsep}{10pt}\framebox{%
\begin{minipage}{0.6\columnwidth}
\tableofcontents
\end{minipage}}
\end{center}


\section{Technology choices}

Paraview supports a wide range of visualizations and supports use in a
distributed environment, as well as supporting parallel processing for
big jobs.  That is, as the engine that creates the visualization, it
appears to be flexible enough to be a visualization server, with many
performance enhancements available at the server end.  This choice was
made before the project began, so we will not revisit it here, but
will discuss the choices made based on that one.

The requirements are to support several basic visualizations:

\begin{enumerate}

\item Mean age using iso-surfaces.  To see iso-surfaces of mean age,
  and how they interact with the vector velocity field.

\item Velocity field streamlines, shaded with scalar values, such as
  the mean age.

\item Simple volume renderings, of such data as mass transfer, or
  oxygenation.

\item Permutations and combinations of the above.
\end{enumerate}


All these visualizations are relatively straightforward to a
sophisticated Paraview user, but they require many steps.  At least as
important are the many options in the Paraview user interface that are
irrelevant to the desired purpose and thus must be ignored.  These
options make for a powerful interface, but a confusing experience.  A
useful application should not merely show data, but should make it
easy for a novice to get satisfying results quickly.

The UI requirements are thus:

\begin{itemize}

\item Should run on lightweight processors, like laptops or tablets,
  maybe even phones.

\item Should guide the user to making interesting views of the data
  they wish to examine.

\item Should preserve screen space in order to be used on small devices.

\end{itemize}

The basic Paraview user interface is heavy and confusing and takes up
a lot of screen space better spent on visualization. The vast majority
of the options available at the top level are not relevant to the
proposed Amgen application.  There are ways to avoid displaying all
the menus, but the subsets that remain when you turn off this menubar
or that one do not match the subset we wish to use.

Paraview supports a ``ParaviewWeb'' interface, and this seems like a
way to address these issues, as well as move the data processing off
the small portable processor.  There are two full interfaces already
built and available: Visualizer and LightViz.  These were evaluated,
and though their design is more modern than the basic Paraview
interface, they are probably not the appropriate solution.  That is,
they seek to preserve all of the flexibility of the original
application, albeit in a friendlier package, and thus provide no
guidance to the user.  Though their interfaces are not as obviously
intimidating as the basic Paraview screen, the options are as
numerous, and there is no guidance available to the user through them.
Learning to use one of these interfaces will take as long as learning
to use the original Paraview interface.

The Paraview server has an optional Python scripting interface
available through the pvpython binary, and this can
serve to ``channel'' the user's choices.  That is, we can create a
small number of Python functions to generate each of the desired
visualizations and control them from the remote web interface that
displays the result.  A modest set of menus can be used to guide the
user through the choices necessary for the visualization.

Julio shared a graphic about how the visualization sequence would
work, and I have edited it slightly to reflect how I think it would
go.  Time runs down in the image, so the process begins with the user
request (RQ) for a model run in the top left.  The shaded boxes
represent the parts of the system that Brown/CCV is responsible for.

\hspace{-.5in}
\includegraphics[width=1.2\columnwidth]{images/agile_visualizations_sequence.pdf}

\vspace{-0.4in}

In the above, the user uses a web browser to request a model run,
and the request is forwarded to some run system and then to a job
manager, per Julio's original description.  Summary data is forwarded
to the user interface for examination.  This part is Amgen's
responsibility.

Subsequent to the generation of the model data, the user initiates the
next step by requesting a visualization.  That RQ is forwarded to the
Paraview server, activating one of a small arsenal of Python functions
pre-loaded into the pvpython server.  That function requests the
relevant model data from the Amgen system infrastructure, wherever it
is, and makes its data available to another function, still within the
pvpython server.  That function in turn generates the desired
visualization of the data.  The Paraview server appears to forward a
2D image of that 3D model
to the ParaviewWeb Javascript infrastructure running in the web
browser, where it is displayed to the user.  The user's manipulation
of the image data appears to interact with the Paraview server while
the user rotates and zooms in and out.  The user will likely iterate
this process, refining
the visualization parameters, until the desired result is achieved.
Only the first RQ for a Paraview model will result in a request for
model data from the :Run module.

\newpage


\begin{wrapfigure}{l}{0.51\columnwidth}
\includegraphics[width=0.5\columnwidth]{images/mockup.pdf}
\end{wrapfigure}

The other important interest in developing this application is to make
the loops shown in the diagram operate as efficiently as possible.
The typical usage of a visualization application involves an iterative
approach to creation: make something, edit one
parameter, change another, and so on.  In the process, it is often
challenging to remember what the options are, as well as remembering
what was actually done to create the view on the screen.

To the left is a crude mockup of a user interface meant to maximize
the screen area used for viewing the model, and an indication of how
the menu system will work.  The idea of the menu structure is to
present the choices to be made across the bottom of the view, in such
a way that when the menus are displayed, they will act as a visible
record of the choices that were actually made to create the view.  The
goal would be to make the choices not only easy to read, but easy to
edit as well. The interface visibility will toggle with the small blue
button to the left and the menus will ``unfold'' from that button.

The ParaviewWeb apparatus relies heavily on the React Javascript UI
framework, so we will use that.  It seems to have tools enough that an
additional UI framework will not be necessary.

\section{How it works}


The documentation for ParaviewWeb is not really adequate, so we
describe the implementation of the server and client here.

\framebox{%
\includegraphics[width=0.8\columnwidth]{images/client-server-diagram.pdf}
}

\subsection{pvpython Server}

\subsubsection{Process communication}

\subsubsection{Protocols}


\subsection{paraviewweb Client}

\subsubsection{Protocol implementation}

\subsubsection{React toolkit}


\end{document}
